\documentclass[12pt,a4paper]{article}

\usepackage[T2A]{fontenc}
\usepackage[utf8]{inputenc}
\usepackage[russian]{babel}

\title{Проектирование структуры данных зоопарка}
\author{Верхотуров В.С. БСБО-05-20\\РТУ МИРЭА}
\date{Март 2022}

\begin{document}

\maketitle

\section{Оценка предметной области}

Зоологический парк - учреждение для содержания животных в неволе с целью их демонстрации, сохранения, воспроизводства и изучения. В задачи зоопарка входят демонстрация разнообразия животного мира, распространение знаний о природе, пропаганда охраны животных и сохранение генофонда редких и исчезающих видов животных. Требуется большая территория и большое количество животных.

В зоопарке происходят взаимодействия между сотрудниками, животными, поставщиками кормов, посетителями.

Сотрудников можно разделить на группы: работники администрации, работники кассы, ветеринары, дрессировщики, заведующие хозяйством. Некоторые имеют прямой контакт с животными.

Для содержания животных необходимо учитывать потребности конкретных видов: климат и размер помещения, вид корма, социальность, прививки, необходимость в помощи с уходом за потомством.

Типы кормов: растительный (хранение в сухом проветриваемом помещении), мясо (хранение в холодильной камере), живой (требуется помещение для содержания, корм). Необходим список поставщиков для сопоставления и выявления наилучшего по цене и качеству поставляемого сырья.

Посетителям требуется удовлетворение их целей: научного, познавательного. Необходим учет контингента, сбор  обратной связи для улучшения предоставляемых услуг.

\end{document}
