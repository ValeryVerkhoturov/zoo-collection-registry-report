\documentclass[10pt, a4paper, titlepage]{article}

\usepackage[T2A]{fontenc}
\usepackage[utf8]{inputenc}
\usepackage[russian]{babel}

\usepackage{hyperref}
\hypersetup{pdftitle={Методология проектирования структур данных. Индивидуальное задание. Разработка модели данных зоопарка}, pdfauthor={Верхотуров В.С.}, colorlinks=false, pdfborder={0 0 0}}

\usepackage{authblk}

\title{Методология проектирования структур данных. Индивидуальное задание. Разработка модели данных зоопарка}
\author{Верхотуров В.С.}
\affil{БСБО-05-20}
\affil{РТУ МИРЭА}
\date\today


\begin{document}

\maketitle

\tableofcontents
\newpage

\section{Анализ предметной области}

\subsection{Общая характеристика}

Зоологический парк - постоянно действующая и открытая для посетителей организация, содержащая на стационарной территории зоологическую коллекцию, включающую диких животных, способствующая сохранению видов животных посредством просвещения, сбора и распространения информации о животных, рекреации и проведения исследований.


\subsection{Организационная структура}

Зоопарки являются:
\begin{itemize}
    \item музеями живой природы, сохраняющими, изучающими и экспонирующими зоологические коллекции живых диких животных различных (в том числе находящихся под угрозой уничтожения) видов;
    
    \item природоохранными организациями, способствующими сохранению видов животных как в неволе, так и в природе и воспитывающими бережное отношение к окружающей среде;
    
    \item культурно-просветительскими центрами, распространяющими е\-сте\-стве\-нно-на\-уч\-ны\-е знания;
    
    \item публичными площадками, предоставляющими возможности проведения культурного досуга различным категориям населения.
\end{itemize}

\subsection{Функциональная структура}


Предоставление Министерству культуры Российской Федерации по установленному зоопарку адресу данных для наблюдения за деятельностью зоопарка по приказу Федеральной службы государственной статистики (Росстат).

\subsection{Характеристика документов}

\href{https://student2.consultant.ru/cgi/online.cgi?req=doc&rnd=z4Vbg&base=LAW&n=394985&dst=100032&field=134#fU2wi2Tgqg0mxQc4}{\textbf{Постановление Правительства РФ от 02.06.2008 N 420 (ред. от 06.09.2021) "О Федеральной службе государственной статистики"}}

Содержание:
\begin{enumerate}
    \item общие положения;
    \item полномочия;
    \item организация деятельности.
\end{enumerate}

Федеральная служба государственной статистики (Росстат) является федеральным органом исполнительной власти, осуществляющим функции по формированию официальной статистической информации о социальных, экономических, демографических, экологических и других общественных процессах в Российской Федерации, а также в порядке и случаях, установленных законодательством Российской Федерации, по контролю в сфере официального статистического учета.

\href{https://student2.consultant.ru/cgi/online.cgi?req=doc&rnd=XbcvLw&base=LAW&n=399590&dst=100013&field=134#2855r1TgPC7pc4QA1}{\textbf{Приказ Росстата от 27.10.2021 N 736 "Об утверждении формы федерального статистического наблюдения с указаниями по ее заполнению для организации Министерством культуры Российской Федерации федерального статистического наблюдения за деятельностью зоопарка (зоосада)"}}

Содержание:
\begin{enumerate}
    \item приказ;
    
    \item форма:
    \begin{enumerate}
        \item общие сведения;
        
        \item научно-просветительная работа за год;
        
        \item численность животных;
        
        \item персонал;
        
        \item поступление и использование финансовых средств;
        
    \end{enumerate}
    
    \item указания.
\end{enumerate}

Приказ об утверждении представленную Министерством культуры Российской Федерации годовую форму федерального статистического наблюдения N 14-НК "Сведения о деятельности зоопарка (зоосада)" с указаниями по ее заполнению (приложение) для сбора и обработки первичных статистических данных в системе Министерства культуры Российской Федерации и ввести ее в действие с отчета за 2021 год.

\section{Модель предметной области}

\subsection{Диаграмма модели предметной области типа <<Сущность -- связь>>}










\subsection{Спецификация модели предметной области}







\subsection{Нормализация схем сущностей}






\section{Реляционная модель данных}




\subsection{Получение реляционных отношений из модели предметной области}

\subsubsection{Таблица перехода}





\subsubsection{Итоговые реляционные отношения}




\subsection{Нормализация реляционных отношений}



\subsection{Спецификация реляционной модели данных}



\end{document}
