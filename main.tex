\documentclass[10pt, a4paper, titlepage]{article}

\usepackage[T2A]{fontenc}
\usepackage[utf8]{inputenc}
\usepackage[russian]{babel}

\usepackage{hyperref}
\hypersetup{pdftitle={Методология проектирования структур данных. Индивидуальное задание. Разработка модели данных зоологической коллекции}, pdfauthor={Верхотуров В.С.}, colorlinks=false, pdfborder={0 0 0}}

\usepackage{authblk} % affilations on the titlepage

\usepackage{graphicx}

\usepackage{float} % Вставить картинку в определенное место

\usepackage{xcolor}

\title{Методология проектирования структур данных. Индивидуальное задание. Разработка модели данных зоологической коллекции}
\author{Верхотуров В.С.}
\affil{БСБО-05-20}
\affil{РТУ МИРЭА}
\date\today


\begin{document}

\maketitle

\tableofcontents
\newpage

\section{Анализ предметной области}

\subsection{Общая характеристика}

Зоологические коллекции (фондовые научные коллекции зоологических институтов, университетов, музеев, а также собрания чучел, препаратов и частей объектов животного мира, живые коллекции зоопарков, зоосадов, цирков, питомников, аквариумов, океанариумов и других учреждений), представляющие научную, культурно-просветительную, учебно-воспитательную и эстетическую ценность, отдельные выдающиеся коллекционные экспонаты независимо от формы их собственности подлежат государственному учету.

\subsection{Организационная структура}
\label{sec:organizationalStructure}

Порядок государственного учета, пополнения, хранения, приобретения, продажи, пересылки, вывоза за пределы Российской Федерации и ввоза в нее зоологических коллекций или отдельных экспонатов определяет уполномоченный Правительством Российской Федерации федеральный орган исполнительной власти.

Государственный учет зоологических коллекций в Российской Федерации осуществляется Государственным комитетом Российской Федерации по охране окружающей среды (Госкомэкология России) путем ведения государственного реестра.

Юридические лица и граждане, являющиеся владельцами таких коллекций и экспонатов, обязаны соблюдать порядок их учета, хранения, использования и пополнения.

\subsection{Функциональная структура}

Предоставление данных Госкомэкологии России для исполнения обязанности по соблюдению порядка учета зоологической коллекции, хранения, использования и пополнения.

\subsection{Характеристика документов}

\href{https://docs.cntd.ru/document/9011346}{\textbf{Федеральный закон от 24 апреля 1995 г. № 52-ФЗ "О животном мире"}}.

Закон содержит раздел <<Статья 29. Зоологические коллекции>>, описанный в \autoref{sec:organizationalStructure}.

Животный мир является достоянием народов Российской Федерации, неотъемлемым элементом природной среды и биологического разнообразия Земли, возобновляющимся природным ресурсом, важным регулирующим и стабилизирующим компонентом биосферы, всемерно охраняемым и рационально используемым для удовлетворения духовных и материальных потребностей граждан Российской Федерации.

\href{https://docs.cntd.ru/document/58812875}{\textbf{Приказ Госкомэкологии России от 30 сентября 1997 г. № 411 "О Положении о зоологических коллекциях" (Минюст N 1507 08.04.98)}}.

Содержание:
\begin{enumerate}
    \item Приказ <<О Положении о зоологических коллекциях>>;
    
    \item Положение о зоологических коллекциях;
    
    \item Форма реестра зоологических коллекций, поставленных на государственный учет;
    
    \item Форма свидетельства о внесении зоологической коллекции в реестр;
    
    \item Форма разрешения на вывоз за пределы Российской Федерации и ввоз на ее территорию зоологических коллекций, их частей и отдельных экспонатов.

\end{enumerate}

Во исполнение постановления \href{https://docs.cntd.ru/document/9026760?marker}{Правительства Российской Федерации от 17.07.96 N 823 "О порядке государственного учета, пополнения, хранения, приобретения, продажи, пересылки, вывоза за пределы Российской Федерации и ввоза на ее территорию зоологических коллекций"} и согласно \href{https://docs.cntd.ru/document/9050595?marker}{Постановлению Правительства Российской Федерации от 26.05.97 N 643 "Об утверждении Положения о Государственном комитете Российской Федерации по охране окружающей среды" (С3, 1997, N 22, ст.2605)}.


\section{Модель предметной области}

\subsection{Диаграмма модели предметной области типа <<Сущность -- связь>>}

\begin{figure}[H]
    \centering
    \includegraphics[height=9cm]{image.png}
    \caption{Диаграмма модели данных зоологической коллекции}
    \label{fig:er_diagram}
\end{figure}

\subsection{Спецификация модели предметной области}

{
    \color{red}
    \textbf{TODO}
}







\subsection{Нормализация схем сущностей}

Отношения находятся в 1 НФ, т.к. кажый его элемент имеет атомарное строение.

Отношения находятся во 2 НФ, т.к. они заданы в 1 НФ и каждый атрибут, не являющийся первичным атрибутом в каждом отношении, функционально полно зависит от первичного ключа этого отношения.

Отношения находятся в 3 НФ, т.к. они заданы во 2 НФ и каждый атрибут  из каждого отношения, не являющийся первичным, нетранзитивно зависит от каждого возможного ключа. 

\section{Реляционная модель данных}




\subsection{Получение реляционных отношений из модели предметной области}

{
    % Macros
    
    %% Keys
    
    %%% Primary Key
    \newcommand{\pk}[1]{\textbf{#1}}
    
    %%% Foreign Key
    \newcommand{\fk}[1]{\textit{#1}}
    
    %%% Primary Foreign Key
    \newcommand{\pfk}[1]{\pk{\fk{#1}}}
    
    %% Columns
    \newcommand{\firstColumn}[4]{#1 - \newline #2 - \newline #3 \newline\newline #4}
    
    \newcommand{\thirdColumn}[6]{
    #1 \newline 
    \underline{Первичный ключ} - #2 \newline 
    \setbox0=\hbox{#3\unskip}\ifdim\wd0=0pt
    \else
      \underline{Внешний(е) ключ(-и)}: #3 \newline
    \fi
    #4 \newline 
    \underline{Первичный ключ} - #5 \newline
    \setbox0=\hbox{#6\unskip}\ifdim\wd0=0pt
    \else
      \underline{Внешний(е) ключ(-и)}: #6 \newline
    \fi
    }
    
    \newcommand\generalizedColumn[6]{\thirdColumn{#1:}{#2}{#3}{#4:}{#5}{#6}}
    
    
    %%  Rules
    
    %%% 1
    \newcommand\ruleOneMondatoryOneMondatoryNum{1}
    \newcommand\ruleOneMondatoryOneMondatory{1 Об - 1 Об}
    
    %%% 2
    \newcommand\ruleOneMondatoryOneOptionalNum{2}
    \newcommand\ruleOneMondatoryOneOptional{1 Об - 1 Н/О}
    \newcommand\ruleOneOptionalOneMondatoryNum{2}
    \newcommand\ruleOneOptionalOneMondatory{1 Н/О - 1 Об}
    
    %%% 3
    \newcommand\ruleOneOptionalOneOptionalNum{3}
    \newcommand\ruleOneOptionalOneOptional{1 Н/О - 1 Н/О}
    
    %%% 4
    \newcommand\ruleOneOptionalManyMondatoryNum{4}
    \newcommand\ruleOneOptionalManyMondatory{1 Н/О - М Об}
    \newcommand\ruleManyMondatoryOneOptionalNum{4}
    \newcommand\ruleManyMondatoryOneOptional{М Об - 1 Н/О}
    
    \newcommand\ruleOneMondatoryManyMondatoryNum{4}
    \newcommand\ruleOneMondatoryManyMondatory{1 Об - М Об}
    \newcommand\ruleManyMondatoryOneMondatoryNum{4}
    \newcommand\ruleManyMondatoryOneMondatory{М Об - 1 Об}
    
    %%% 5
    \newcommand\ruleOneMondatoryManyOptionalNum{5}
    \newcommand\ruleOneMondatoryManyOptional{1 Об - М Н/О}
    \newcommand\ruleManyOptionalOneMondatoryNum{5}
    \newcommand\ruleManyOptionalOneMondatory{М Н/О - 1 Об}
    
    \newcommand\ruleOneOptionalManyOptionalNum{5}
    \newcommand\ruleOneOptionalManyOptional{1 Н/О - М Н/О}
    \newcommand\ruleManyOptionalOneOptionalNum{5}
    \newcommand\ruleManyOptionalOneOptional{М Н/О - 1 Н/О}
    
    %%% 6
    \newcommand\ruleManyOptionalManyOptionalNum{6}
    \newcommand\ruleManyOptionalManyOptional{М Н/О - М Н/О}
    
    %% Entities
    
    %%% Зоологическая коллекция
    \newcommand\zoologicheskayaKollektsiya{Зоологическая коллекция (\pk{Номер госрегистрации}, Дата регистрации, Фамилия владельца, Имя владельца, Отчество владельца, \fk{Код ФИАС владельца}, \fk{Код ФИАС местонахождения коллекции}, \fk{Код формы собственности}, Год основания коллекции, Номер перерегистрации, \fk{Код заключения Экспертной группы по зоологическим коллекциям}, Фамилия выдавшего свидетельство о внесении коллекции в реестр, Имя выдавшего свидетельство о внесении коллекции в реестр, Отчество выдавшего свидетельство о внесении коллекции в реестр, \fk{Код ОКПРТР выдавшего свидетельство о внесении коллекции в реестр}, Номер свидетельства о внесении коллекции в реестр)}
    
    \newcommand\zoologicheskayaKollektsiyaPK{Номер госрегистрации}
    \newcommand\zoologicheskayaKollektsiyaFK{Код ФИАС владельца, Код ФИАС местонахождения коллекции, Код формы собственности, Код заключения Экспертной группы по зоологическим коллекциям, Код заключения Экспертной группы по зоологическим коллекциям, Код ОКПРТР выдавшего свидетельство о внесении коллекции в реестр}
    
    %%% Классификатор ФИАС
    \newcommand\krFIAS{Классификатор ФИАС (\pk{Номер ФИАС}, Наименование ФИАС)}
    
    \newcommand\krFIASPK{Номер ФИАС}
    \newcommand\krFIASFK{}
    
    %%% Классификатор форм собственности
    \newcommand\krFormSobstvennosti{Классификатор форм собственности (\pk{Код формы собственности}, Наименование формы собственности)}
    
    \newcommand\krFormSobstvennostiPK{Код формы собственности}
    \newcommand\krFormSobstvennostiFK{}
    
    %%% Классификатор назначения коллекции
    \newcommand\krNaznacheniyaKollektsii{Классификатор назначения коллекции (\pk{Код назначения коллекции}, Наименование назначения коллекции)}
    
    \newcommand\krNaznacheniyaKollektsiiPK{Код назначения коллекции}
    \newcommand\krNaznacheniyaKollektsiiFK{}
    
    %%% Объекты хранения
    \newcommand\obekiHranenia{Объекты хранения (\pfk{Номер госрегистрации}, \fk{Код зоологической номенклатуры}, Количество объектов хранения, Является уникальным объектом)}
    
    \newcommand\obekiHraneniaPK{Номер госрегистрации}
    \newcommand\obekiHraneniaFK{Номер госрегистрации, Код зоологической номенклатуры}

    %%% Зоологическая номенклатура
    \newcommand\zoologicheskayaNpmenklatura{Зоологическая номенклатура (\pk{Код зоологической номенклатуры}, Наименование таксона)}
    
    \newcommand\zoologicheskayaNpmenklaturaPK{Код зоологической номенклатуры}
    \newcommand\zoologicheskayaNpmenklaturaFK{}
    
    %%% Происхождение коллекции
    \newcommand\proishozhdeniyeKollektsii{Происхождение коллекции (\pfk{Номер госрегистрации}, \pk{Номер документа}, \fk{Код вида документа}, Серия, Номер)}
    
    \newcommand\proishozhdeniyeKollektsiiPK{Номер госрегистрации, Номер документа}
    \newcommand\proishozhdeniyeKollektsiiFK{Номер госрегистрации, Код вида документа}
    
    %%% Классификатор вида документов
    \newcommand\krVidaDokumentov{Классификатор вида документов (\pk{Код вида документов}, Наименование вида документа)}
    
    \newcommand\krVidaDokumentovPK{Код вида документов}
    \newcommand\krVidaDokumentovFK{}
    
    %%% Описание коллекции
    \newcommand\opisaniyeKollektsii{Описание коллекции (\pfk{Номер госрегистрации}, \pk{Номер описания}, \fk{Код описания коллекции})}
    
    \newcommand\opisaniyeKollektsiiPK{Номер госрегистрации, Номер описания}
    \newcommand\opisaniyeKollektsiiFK{Номер госрегистрации, Код описания коллекции}
    
    %%% Классификатор коллекций
    \newcommand\krKollektsiy{Классификатор коллекций (\pk{Код описания коллекции}, Наименование описания коллекции)}
    
    \newcommand\krKollektsiyPK{Код описания коллекции}
    \newcommand\krKollektsiyFK{}
    
    %%% Изменения в составе коллекции
    \newcommand\izmeneniaVSostaveKollektsii{Изменения в составе коллекции (\pfk{Код госрегистрации}, \pk{Номер изменения}, \fk{Код причины изменения}, Описание измения)}
    
    \newcommand\izmeneniaVSostaveKollektsiiPK{Код госрегистрации, Номер изменения}
    \newcommand\izmeneniaVSostaveKollektsiiFK{Код госрегистрации, Код причины изменения}
    
    %%% Классификатор причин изменения в составе коллекции
    \newcommand\krPrichinIzmeneniaVSostaveKollektsii{Классификатор причин изменения в составе коллекции (\pk{Код причины именения}, Наименование причины изменения)}
    
    \newcommand\izmeneniaVSostaveKollektsiiPK{Код причины именения}
    \newcommand\izmeneniaVSostaveKollektsiiFK{}
    
    %%% Условия хранения коллекции
    \newcommand\usoviaHraneniaKollektsii{Условия хранения коллекции (\pfk{Номер госрегистрации}, \pk{Номер условия}, \fk{Код условия хранения})}
    
    \newcommand\usoviaHraneniaKollektsiiPK{Номер госрегистрации, Номер условия}
    \newcommand\usoviaHraneniaKollektsiiFK{Номер госрегистрации, Код условия хранения}
    
    %%% Классификатор условий хранения
    \newcommand\krUsloviyHranenia{Классификатор условий хранения (\pk{Код условия хранения}, Наименование условия хранения)}
    
    \newcommand\krUsloviyHraneniaPK{Код условия хранения}
    \newcommand\krUsloviyHraneniaFK{}
    
    %%% Классификатор Экспертной группы по зоологическим коллекциям
    \newcommand\krEkspertnoiGruppiPoZoologicheskimKolleksiyam{Классификатор Экспертной группы по зоологическим коллекциям (\pk{Код заключения}, Наименование заключения)}
    
    \newcommand\krEkspertnoiGruppiPoZoologicheskimKolleksiyamPK{Код заключения}
    \newcommand\krEkspertnoiGruppiPoZoologicheskimKolleksiyamFK{}
    
    %%% Классификатор ОКПРТР
    \newcommand\krOKPRTR{Классификатор ОКПРТР (\pk{Код ОКПРТР}, Наименование ОКПРТР)}
    
    \newcommand\krOKPRTRPK{Код ОКПРТР}
    \newcommand\krOKPRTRFK{}
    
    %%% Классификатор ОКПРТР
    \newcommand\krOKPRTR{Классификатор ОКПРТР (\pk{Код ОКПРТР}, Наименование ОКПРТР)}
    
    \newcommand\krOKPRTRPK{Код ОКПРТР}
    \newcommand\krOKPRTRFK{}
    
    %%% Разрешение на вывоз и ввоз
    \newcommand\razresheniyeNaVivozIVvoz{Разрешение на вывоз и ввоз (Действителен с, Действителен по, \pk{Номер разрешения}, \pfk{Номер госрегистрации}, \fk{Код ОКСМ ввоза}, \fk{Код ОКСМ вывоза}, Фамилия отправителя, Имя отправителя, Отчество отправителя, Фамилия получателя, Имя получателя, Отчество получателя, \fk{Код цели ввоза или вывоза}, Дата выдачи разрешения, Фамилия выдавшего разрешение, Имя выдавшего разрешение, Отчество выдавшего разрешение, \fk{Код ОКПРТР выдавшего разрешение})}
    
    \newcommand\razresheniyeNaVivozIVvozPK{Номер разрешения, Номер госрегистрации}
    \newcommand\razresheniyeNaVivozIVvozFK{Номер госрегистрации, Код ОКСМ ввоза, Код ОКСМ вывоза, Код цели ввоза или вывоза, Код ОКПРТР выдавшего разрешение}
    
    %%% Классификатор целей ввоза и вывоза
    \newcommand\krTseleyVvozaIVivoza{Классификатор целей ввоза и вывоза (\pk{Код цели ввоза или вывоза}, Наименование цели ввоза или вывоза)}
    
    \newcommand\krTseleyVvozaIVivozaPK{Код цели ввоза или вывоза}
    \newcommand\krTseleyVvozaIVivozaFK{}
    
    %%% Особые условия ввоза и вывоза
    \newcommand\osbiyeUsloviaVvozaIVivoza{Особые условия ввоза и вывоза (\pfk{Номер разрешения ввоза или вывоза}, \pfk{Номер госрегистрации}, \pk{Номер особого условия ввоза или вывоза}, \fk{Код особого условия ввоза или вывоза})}
    
    \newcommand\osbiyeUsloviaVvozaIVivozaPK{Номер разрешения ввоза или вывоза, Номер госрегистрации, Номер особого условия ввоза или вывоза}
    \newcommand\osbiyeUsloviaVvozaIVivozaFK{Номер разрешения ввоза или вывоза, Номер госрегистрации, Код особого условия ввоза или вывоза}
    
    %%% Классификатор особых условий ввоза и вывоза
    \newcommand\krTseleyVvozaIVivoza{Классификатор особых условий ввоза и вывоза (\pk{Код особого условия}, Наименование осбого условия ввоза или вывоза)}
    
    \newcommand\krTseleyVvozaIVivozaPK{Код особого условия}
    \newcommand\krTseleyVvozaIVivozaFK{}
    
    
    \subsubsection{Таблица перехода}
    
    
    
    
    
    
    \subsubsection{Итоговые реляционные отношения}
    
    
    
    
}

\subsection{Нормализация реляционных отношений}



\subsection{Спецификация реляционной модели данных}



\end{document}
