\documentclass[10pt, a4paper, titlepage]{article}

\usepackage[T2A]{fontenc}
\usepackage[utf8]{inputenc}
\usepackage[russian]{babel}

\usepackage{hyperref}
\hypersetup{pdftitle={Методология проектирования структур данных. Индивидуальное задание. Разработка модели данных зоологической коллекции}, pdfauthor={Верхотуров В.С.}, colorlinks=false, pdfborder={0 0 0}}

\usepackage{authblk} % affilations on the titlepage

\title{Методология проектирования структур данных. Индивидуальное задание. Разработка модели данных зоологической коллекции}
\author{Верхотуров В.С.}
\affil{БСБО-05-20}
\affil{РТУ МИРЭА}
\date\today


\begin{document}

\maketitle

\tableofcontents
\newpage

\section{Анализ предметной области}

\subsection{Общая характеристика}

Зоологические коллекции (фондовые научные коллекции зоологических институтов, университетов, музеев, а также собрания чучел, препаратов и частей объектов животного мира, живые коллекции зоопарков, зоосадов, цирков, питомников, аквариумов, океанариумов и других учреждений), представляющие научную, культурно-просветительную, учебно-воспитательную и эстетическую ценность, отдельные выдающиеся коллекционные экспонаты независимо от формы их собственности подлежат государственному учету.

\subsection{Организационная структура}

Порядок государственного учета, пополнения, хранения, приобретения, продажи, пересылки, вывоза за пределы Российской Федерации и ввоза в нее зоологических коллекций или отдельных экспонатов определяет уполномоченный Правительством Российской Федерации федеральный орган исполнительной власти.

Государственный учет зоологических коллекций в Российской Федерации осуществляется Государственным комитетом Российской Федерации по охране окружающей среды (Госкомэкология России) путем ведения государственного реестра.

Юридические лица и граждане, являющиеся владельцами таких коллекций и экспонатов, обязаны соблюдать порядок их учета, хранения, использования и пополнения.

\subsection{Функциональная структура}

Предоставление данных Госкомэкологии России для исполнения обязанности соблюдения порядка учета зоологической коллекции, хранения, использования и пополнения.

\subsection{Характеристика документов}

\href{https://docs.cntd.ru/document/9011346}{\textbf{Федеральный закон от 24 апреля 1995 г. № 52-ФЗ "О животном мире".}}

Содержит\footnote{разделы, относящиеся к предметной области}:
\begin{enumerate}
    \item Статья 29. Зоологические коллекции.
\end{enumerate}

Животный мир является достоянием народов Российской Федерации, неотъемлемым элементом природной среды и биологического разнообразия Земли, возобновляющимся природным ресурсом, важным регулирующим и стабилизирующим компонентом биосферы, всемерно охраняемым и рационально используемым для удовлетворения духовных и материальных потребностей граждан Российской Федерации.

\href{https://docs.cntd.ru/document/58812875}{\textbf{Приказ Госкомэкологии России от 30 сентября 1997 г. № 411 "О Положении о зоологических коллекциях" (Минюст N 1507 08.04.98)}}

Содержит:
\begin{enumerate}
    \item Приказ <<О Положении о зоологических коллекциях>>;
    
    \item Положение о зоологических коллекциях;
    
    \item Форма реестра зоологических коллекций, поставленных на государственный учет;
    
    \item Форма свидетельства о внесении зоологической коллекции в реестр;
    
    \item Форма разрешения на вывоз за пределы Российской Федерации и ввоз на ее территорию зоологических коллекций, их частей и отдельных экспонатов.

\end{enumerate}

Во исполнение постановления \href{https://docs.cntd.ru/document/9026760?marker}{Правительства Российской Федерации от 17.07.96 N 823 "О порядке государственного учета, пополнения, хранения, приобретения, продажи, пересылки, вывоза за пределы Российской Федерации и ввоза на ее территорию зоологических коллекций"} и согласно \href{https://docs.cntd.ru/document/9050595?marker}{Постановлению Правительства Российской Федерации от 26.05.97 N 643 "Об утверждении Положения о Государственном комитете Российской Федерации по охране окружающей среды" (С3, 1997, N 22, ст.2605)}


\section{Модель предметной области}

\subsection{Диаграмма модели предметной области типа <<Сущность -- связь>>}











\subsection{Спецификация модели предметной области}







\subsection{Нормализация схем сущностей}






\section{Реляционная модель данных}




\subsection{Получение реляционных отношений из модели предметной области}

\subsubsection{Таблица перехода}





\subsubsection{Итоговые реляционные отношения}




\subsection{Нормализация реляционных отношений}



\subsection{Спецификация реляционной модели данных}



\end{document}
