\documentclass[12pt, a4paper]{article}

\usepackage[T2A]{fontenc}
\usepackage[utf8]{inputenc}
\usepackage[russian]{babel}

\usepackage{hyperref}
\hypersetup{pdftitle={Проектирование структуры данных зоопарка}, pdfauthor={Верхотуров В.С.}, colorlinks=false, pdfborder={0 0 0}}

\usepackage[affil-it]{authblk}

\title{Проектирование структуры данных зоопарка}
\author{Верхотуров В.С. БСБО-05-20}
\affil{РТУ МИРЭА}
\date{\today}


\begin{document}

\maketitle

\section{Краткая характеристика предметной области}

Зоологический парк - постоянно действующая и открытая для посетителей организация, содержащая на стационарной территории зоологическую коллекцию, включающую диких животных, способствующая сохранению видов животных посредством просвещения, сбора и распространения информации о животных, рекреации и проведения исследований.


\section{Цель предметной области}

Зоопарки являются:
\begin{itemize}
    \item музеями живой природы, сохраняющими, изучающими и экспонирующими зоологические коллекции живых диких животных различных (в том числе находящихся под угрозой уничтожения) видов;
    
    \item природоохранными организациями, способствующими сохранению видов животных как в неволе, так и в природе и воспитывающими бережное отношение к окружающей среде;
    
    \item культурно-просветительскими центрами, распространяющими е\-сте\-стве\-нно-на\-уч\-ны\-е знания;
    
    \item публичными площадками, предоставляющими возможности проведения культурного досуга различным категориям населения.
\end{itemize}

\section{Функции предметной области}

К оптимальному набору функций зоопарка относят: природоохранную, исследовательскую, просветительскую, воспитательную и рекреационную функции.

\section{Нормативные правовые документы предметной области}

\begin{itemize}
    \item \href{https://student2.consultant.ru/cgi/online.cgi?req=doc&base=LAW&n=373488&dst=0#Gf5ur1TPkK5LgmbH}{Закон РФ от 07.02.1992 N 2300-1 (ред. от 11.06.2021) "О защите прав потребителей"};
    
    \item \href{https://docs.cntd.ru/document/1200137228}{"ГОСТ Р 57013-2016. Национальный стандарт Российской Федерации. Услуги населению. Услуги зоопарков. Общие требования" (утв. и введен в действие Приказом Росстандарта от 20.07.2016 N 857-ст)};
    
    \item \href{https://student2.consultant.ru/cgi/online.cgi?req=doc&rnd=XbcvLw&base=LAW&n=399590&dst=100013&field=134#2855r1TgPC7pc4QA1}{Приказ Росстата от 27.10.2021 N 736 "Об утверждении формы федерального статистического наблюдения с указаниями по ее заполнению для организации Министерством культуры Российской Федерации федерального статистического наблюдения за деятельностью зоопарка (зоосада)"}.
\end{itemize}


\end{document}

%Зоологический парк - учреждение для содержания животных в неволе с целью их демонстрации, сохранения, воспроизводства и изучения. В задачи зоопарка входят демонстрация разнообразия животного мира, распространение знаний о природе, пропаганда охраны животных и сохранение генофонда редких и исчезающих видов животных. Требуется большая территория и большое количество животных.

%В зоопарке происходят взаимодействия между сотрудниками, животными, поставщиками кормов, посетителями.

%Сотрудников можно разделить на группы: работники администрации, работники кассы, ветеринары, дрессировщики, заведующие хозяйством. Некоторые имеют прямой контакт с животными.

%Для содержания животных необходимо учитывать потребности конкретных видов: климат и размер помещения, вид корма, социальность, прививки, необходимость в помощи с уходом за потомством.

%Типы кормов: растительный (хранение в сухом проветриваемом помещении), мясо (хранение в холодильной камере), живой (требуется помещение для содержания, корм). Необходим список поставщиков для сопоставления и выявления наилучшего по цене и качеству поставляемого сырья.

%Посетителям требуется удовлетворение их целей: научного, познавательного. Необходим учет контингента, сбор  обратной связи для улучшения предоставляемых услуг.
